% Options for packages loaded elsewhere
\PassOptionsToPackage{unicode}{hyperref}
\PassOptionsToPackage{hyphens}{url}
%
\documentclass[
]{article}
\usepackage{amsmath,amssymb}
\usepackage{lmodern}
\usepackage{iftex}
\ifPDFTeX
  \usepackage[T1]{fontenc}
  \usepackage[utf8]{inputenc}
  \usepackage{textcomp} % provide euro and other symbols
\else % if luatex or xetex
  \usepackage{unicode-math}
  \defaultfontfeatures{Scale=MatchLowercase}
  \defaultfontfeatures[\rmfamily]{Ligatures=TeX,Scale=1}
\fi
% Use upquote if available, for straight quotes in verbatim environments
\IfFileExists{upquote.sty}{\usepackage{upquote}}{}
\IfFileExists{microtype.sty}{% use microtype if available
  \usepackage[]{microtype}
  \UseMicrotypeSet[protrusion]{basicmath} % disable protrusion for tt fonts
}{}
\makeatletter
\@ifundefined{KOMAClassName}{% if non-KOMA class
  \IfFileExists{parskip.sty}{%
    \usepackage{parskip}
  }{% else
    \setlength{\parindent}{0pt}
    \setlength{\parskip}{6pt plus 2pt minus 1pt}}
}{% if KOMA class
  \KOMAoptions{parskip=half}}
\makeatother
\usepackage{xcolor}
\usepackage[margin=1in]{geometry}
\usepackage{graphicx}
\makeatletter
\def\maxwidth{\ifdim\Gin@nat@width>\linewidth\linewidth\else\Gin@nat@width\fi}
\def\maxheight{\ifdim\Gin@nat@height>\textheight\textheight\else\Gin@nat@height\fi}
\makeatother
% Scale images if necessary, so that they will not overflow the page
% margins by default, and it is still possible to overwrite the defaults
% using explicit options in \includegraphics[width, height, ...]{}
\setkeys{Gin}{width=\maxwidth,height=\maxheight,keepaspectratio}
% Set default figure placement to htbp
\makeatletter
\def\fps@figure{htbp}
\makeatother
\setlength{\emergencystretch}{3em} % prevent overfull lines
\providecommand{\tightlist}{%
  \setlength{\itemsep}{0pt}\setlength{\parskip}{0pt}}
\setcounter{secnumdepth}{-\maxdimen} % remove section numbering
\ifLuaTeX
  \usepackage{selnolig}  % disable illegal ligatures
\fi
\IfFileExists{bookmark.sty}{\usepackage{bookmark}}{\usepackage{hyperref}}
\IfFileExists{xurl.sty}{\usepackage{xurl}}{} % add URL line breaks if available
\urlstyle{same} % disable monospaced font for URLs
\hypersetup{
  hidelinks,
  pdfcreator={LaTeX via pandoc}}

\author{}
\date{\vspace{-2.5em}}

\begin{document}

\hypertarget{assignment1}{%
\section{Assignment1}\label{assignment1}}

\hypertarget{section}{%
\subsection{1}\label{section}}

\begin{enumerate}
\def\labelenumi{\arabic{enumi}.}
\item
  One-way Propagation Delay: The propagation delay \(d\) is calculated
  by the formula: \[
  d = \frac{Distance}{Speed~of~Light}
  \]
\item
  Transmission Time: The time \(t\) it takes to transmit the data at a
  given rate is given by: \[
  t = \frac{Size~of~Data}{Transmission~Rate}
  \]
\item
  Total Time for data to be received on Earth: \[
  Total~Time = Propagation~Delay + Transmission~Time
  \]
\end{enumerate}

\hypertarget{a-one-way-propagation-delay}{%
\subsubsection{(a) One-way Propagation
Delay}\label{a-one-way-propagation-delay}}

\begin{itemize}
\tightlist
\item
  Distance between Earth and Mars at closest approach =
  \(62.07 \times 10^{9}\)meters (62.07 Gm)
\item
  Speed of light = \(3 \times 10^{8}\)meters/second \[
  Propagation~Delay = \frac{62.07 \times 10^{9}}{3 \times 10^{8}} 
  \] \[
  Propagation~Delay \approx 206.9~seconds
  \]
\end{itemize}

\hypertarget{b-time-of-picture-receipt-on-earth}{%
\subsubsection{(b) Time of Picture Receipt on
Earth}\label{b-time-of-picture-receipt-on-earth}}

\begin{itemize}
\tightlist
\item
  Image size = 4Mb = \(4 \times 10^{6}\)bits
\item
  Transmission rate = 256kbps = \(256 \times 10^{3}\)bits/second \[
  Transmission~Time = \frac{4 \times 10^{6}}{256 \times 10^{3}} 
  \] \[
  Transmission~Time = 15.625~seconds
  \]
\end{itemize}

Total Time for the picture to be received on Earth would be the sum of
the propagation delay and the transmission time:

\[
Total~Time = 206.9 + 15.625
\] \[
Total~Time = 222.525~seconds
\]

So, at the time of closest approach, the image from the rover would be
received on Earth in approximately 222.525 seconds after being sent.

\hypertarget{section-1}{%
\subsection{2}\label{section-1}}

\begin{enumerate}
\def\labelenumi{\arabic{enumi}.}
\tightlist
\item
  Queuing Delay: Queuing delay is the time a packet waits in the queue
  before it can be transmitted.
\item
  Transmission Time: \(t\) is given
  by:\[t = \frac{{\text{Size of Data}}}{{\text{Transmission Rate}}}\]
\item
  Maximum Queuing Delay: This is the delay experienced by the last
  packet in the queue.
\item
  Minimum Queuing Delay: This is the delay experienced by the first
  packet in the queue. If the outgoing link is free, this is typically
  zero.
\item
  Average Queuing Delay: This is the average of the queuing delays
  experienced by all the packets.
\end{enumerate}

\hypertarget{given}{%
\subsubsection{Given}\label{given}}

\begin{itemize}
\tightlist
\item
  Size of each packet = 3200 bytes = \(3200 \times 8\)bits = \(25600\)
  bits
\item
  Number of packets = 10
\item
  Transmission rate = 20 Mbps = \(20 \times 10^6\)bits/second
\end{itemize}

\hypertarget{a-maximum-queuing-delay}{%
\subsubsection{(a) Maximum Queuing
Delay}\label{a-maximum-queuing-delay}}

To find the maximum queuing delay, we must find the time it takes to
transmit all packets before the last one.

Transmission time for one packet: \[
t = \frac{{25600}}{{20 \times 10^6}} = \frac{{25600}}{{2 \times 10^7}} = \frac{{128}}{{10^6}} = 0.128 \text{ seconds} = 128 \text{ ms}
\]

Maximum Queuing Delay for the last packet: \[
\text{Max Queuing Delay} = (10 - 1) \times 128 \text{ ms} = 9 \times 128 \text{ ms} = 1152 \text{ ms}
\]

\hypertarget{b-minimum-queuing-delay}{%
\subsubsection{(b) Minimum Queuing
Delay}\label{b-minimum-queuing-delay}}

The minimum queuing delay would be for the first packet. If the outgoing
link is free (which is given), the minimum queuing delay is zero. \[
\text{Min Queuing Delay} = 0 \text{ ms}
\]

\hypertarget{c-average-queuing-delay}{%
\subsubsection{\texorpdfstring{\(c\) Average Queuing
Delay}{c Average Queuing Delay}}\label{c-average-queuing-delay}}

The average queuing delay is the average of the delays experienced by
all the packets. It's the sum of the queuing delays divided by the
number of packets.

\[
\text{Average Queuing Delay} = \frac{{0 + 128 + 256 + \ldots + 1152}}{10}
\] \[
\text{Average Queuing Delay} = \frac{{(9 \times 10 / 2) \times 128}}{10} = \frac{{9 \times 128}}{2} = \frac{{1152}}{2} = 576 \text{ ms}
\]

\hypertarget{d-queue-limitations}{%
\subsubsection{(d) Queue Limitations}\label{d-queue-limitations}}

If the queue can only hold 6 packets, then the maximum queuing delay
would be for the last packet in the limited queue.

\[
\text{Max Queuing Delay with Limited Queue} = (6 - 1) \times 128 \text{ ms} = 5 \times 128 \text{ ms} = 640 \text{ ms}
\]

For the other 4 packets that cannot be held in the queue, they would
typically be dropped or sent to a different queue, depending on the
router's packet dropping or buffering policy.

\hypertarget{section-2}{%
\subsection{3}\label{section-2}}

\hypertarget{a-throughput-with-original-rates}{%
\subsubsection{(a) Throughput with Original
Rates}\label{a-throughput-with-original-rates}}

The throughput will be determined by the link with the lowest rate (the
bottleneck). The rates given are:

\begin{itemize}
\tightlist
\item
  \(R1 = 500\) kbps
\item
  \(R2 = 2\) Mbps
\item
  \(R3 = 1\) Mbps
\end{itemize}

The bottleneck is \(R1 = 500\) kbps. So the throughput for the file
transfer is \(500\) kbps.

\hypertarget{b-time-to-transfer-file-with-original-rates}{%
\subsubsection{(b) Time to Transfer File with Original
Rates}\label{b-time-to-transfer-file-with-original-rates}}

The file size is \(4\) million bytes. First, let's convert this size
into bits: \[
\text{File Size} = 4 \times 10^6 \text{ bytes} \times 8 \text{ bits/byte} = 32 \times 10^6 \text{ bits} = 32 \text{ Mbits}
\]

The time \(t\) to transfer this file at \(500\) kbps is: \[
t = \frac{{32 \text{ Mbits}}}{{500 \text{ kbps}}} = \frac{{32 \times 10^6 \text{ bits}}}{{500 \times 10^3 \text{ bits/sec}}} = \frac{{32 \times 10^6}}{{500 \times 10^3}} \text{ sec} = 64 \text{ sec}
\]

\hypertarget{c-throughput-with-reduced-r2}{%
\subsubsection{\texorpdfstring{\(c\) Throughput with Reduced
\(R2\)}{c Throughput with Reduced R2}}\label{c-throughput-with-reduced-r2}}

Now \(R2\) is reduced to \(100\) kbps, which becomes the new bottleneck.
So the throughput for the file transfer is now \(100\) kbps. The time
\(t\) to transfer this file at \(100\) kbps is: \[
t = \frac{{32 \text{ Mbits}}}{{100 \text{ kbps}}} = \frac{{32 \times 10^6 \text{ bits}}}{{100 \times 10^3 \text{ bits/sec}}} = \frac{{32 \times 10^6}}{{100 \times 10^3}} \text{ sec} = 320 \text{ sec}
\]

So with \(R2\) reduced to \(100\) kbps:

\begin{enumerate}
\def\labelenumi{\arabic{enumi}.}
\tightlist
\item
  The throughput becomes \(100\) kbps.
\item
  The time to transfer the \(4\) million byte file becomes \(320\)
  seconds.
\end{enumerate}

\hypertarget{section-3}{%
\subsection{4}\label{section-3}}

\hypertarget{given-1}{%
\subsubsection{Given:}\label{given-1}}

\begin{itemize}
\tightlist
\item
  Base HTML file size: 5 KBytes = \(5 \times 8 \times 1024\) bits =
  \(40,960\) bits
\item
  Each referenced object: 200 KBytes = \(200 \times 8 \times 1024\) bits
  =\(1,638,400\) bits
\item
  Each control message: 200 bits
\item
  Transmission rate: 10 Mbps = \(10 \times 10^6\) bits/sec
\item
  One-way propagation delay: 50 ms = \(0.05\) sec
\item
  Number of referenced objects: 8
\end{itemize}

\hypertarget{definitions-and-formulas}{%
\subsubsection{Definitions and
Formulas:}\label{definitions-and-formulas}}

\begin{enumerate}
\def\labelenumi{\arabic{enumi}.}
\tightlist
\item
  Round-trip time (RTT) = \(2 \times \text{Propagation delay}\)
\item
  Transmission time =
  \(\frac{\text{File Size}}{\text{Transmission Rate}}\)
\item
  Total time for each method will depend on various factors such as
  connection establishment, transmission times, and RTT.
\end{enumerate}

\hypertarget{common-calculations}{%
\subsubsection{Common Calculations:}\label{common-calculations}}

\begin{enumerate}
\def\labelenumi{\arabic{enumi}.}
\tightlist
\item
  RTT = \(2 \times 0.05\) sec = \(0.1\)sec
\item
  Transmission time for the base HTML file =
  \(\frac{40,960}{10 \times 10^6}\)sec = \(0.004096\) sec
\item
  Transmission time for one referenced object =
  \(\frac{1,638,400}{10 \times 10^6}\) sec = \(0.16384\) sec
\end{enumerate}

\hypertarget{a-using-basic-non-persistent-http-with-no-parallel-connections}{%
\subsubsection{(a) Using basic non-persistent HTTP with no parallel
connections}\label{a-using-basic-non-persistent-http-with-no-parallel-connections}}

For each object (including the base HTML file), we need to establish a
new connection, send a request, and then receive the object. This
process involves:

\begin{itemize}
\tightlist
\item
  1 RTT for the TCP handshake
\item
  1 RTT for the HTTP request and response
\end{itemize}

For the base HTML file: \[
1 \text{ RTT (handshake)} + 1 \text{ RTT (HTTP)} + 0.004096 \text{ sec (transmission)}
= 0.1 + 0.1 + 0.004096 = 0.204096 \text{ sec}
\]

For each referenced object (8 in total): \[
1 \text{ RTT (handshake)} + 1 \text{ RTT (HTTP)} + 0.16384 \text{ sec (transmission)}= 0.1 + 0.1 + 0.16384 = 0.36384 \text{ sec}
\]

Total time: \[
0.204096 \text{ sec (base)} + 8 \times 0.36384 \text{ sec (each object)} = 0.204096 + 2.91072 = 3.114816 \text{ sec}
\]

\hypertarget{b-using-non-persistent-http-with-parallel-connections}{%
\subsubsection{(b) Using non-persistent HTTP with parallel
connections}\label{b-using-non-persistent-http-with-parallel-connections}}

The bandwidth is shared, so each of the 8 connections will receive
\(1/8\) of the 10 Mbps bandwidth, or \(1.25\) Mbps.

New transmission time for one referenced object =
\(\frac{1,638,400}{1.25 \times 10^6}\)sec = \(1.31072\) sec

Total time:

\[
0.204096 \text{ sec (base)} + 0.1 \text{ sec (handshake for first object)} + 0.1 \text{ sec (HTTP request for first object)} + 1.31072 \text{ sec (transmission for one object)}= 0.204096 + 0.1 + 0.1 + 1.31072 = 1.714816 \text{ sec}
\]

\hypertarget{c-using-persistent-http-non-pipelined-no-parallel-connections}{%
\subsubsection{\texorpdfstring{\(c\) Using persistent HTTP
(non-pipelined, no parallel
connections)}{c Using persistent HTTP (non-pipelined, no parallel connections)}}\label{c-using-persistent-http-non-pipelined-no-parallel-connections}}

In persistent HTTP, one connection is established and kept open.

Total time:

\[
0.204096 \text{ sec (base)} + 8 \times (1 \text{ RTT (HTTP) } + 0.16384 \text{ sec (transmission)})= 0.204096 + 8 \times (0.1 + 0.16384) = 0.204096 + 8 \times 0.26384 = 0.204096 + 2.11072 = 2.314816 \text{ sec}
\]

In summary:

\begin{itemize}
\tightlist
\item
  For basic non-persistent HTTP with no parallel connections, it will
  take \(3.114816\) sec
\item
  For non-persistent HTTP with parallel connections, it will take
  \(1.714816\) sec
\item
  For persistent HTTP (non-pipelined, no parallel connections), it will
  take \(2.314816\) sec
\end{itemize}

\hypertarget{section-4}{%
\subsection{5}\label{section-4}}

When you enter a URL in your web browser and the IP address associated
with that URL is not cached on your local host, a DNS (Domain Name
System) lookup will be initiated to resolve the URL into an IP address.
Assuming that no caching has occurred, the following DNS servers will
typically be involved:

\begin{enumerate}
\def\labelenumi{\arabic{enumi}.}
\item
  \textbf{Local DNS Server (Recursive Resolver)}: Your browser will
  first query the local DNS server. If it doesn't have the information,
  it will start a series of queries to find the correct IP address. This
  is usually the DNS server of your Internet Service Provider (ISP), or
  it could be a public DNS server like Google's 8.8.8.8.
\item
  \textbf{Root DNS Server}: The local DNS server queries the root DNS
  server if it doesn't have the IP address in its cache. The root server
  returns a reference to a TLD (Top-Level Domain) DNS server.
\item
  \textbf{TLD DNS Server}: The local DNS server then queries the TLD DNS
  server for the domain. For example, for \texttt{www.example.com}, this
  would be the \texttt{.com} TLD DNS server. The TLD server returns a
  reference to an Authoritative DNS server that has the actual IP
  address.
\item
  \textbf{Authoritative DNS Server}: Finally, the local DNS server
  queries the authoritative DNS server for the actual IP address of the
  domain. Once it gets the IP address, it returns it to your browser.
\end{enumerate}

\hypertarget{minimum-number-of-dns-queries}{%
\subsubsection{Minimum Number of DNS
Queries}\label{minimum-number-of-dns-queries}}

\begin{enumerate}
\def\labelenumi{\arabic{enumi}.}
\tightlist
\item
  One query to the local DNS server, which might involve:

  \begin{enumerate}
  \def\labelenumii{\arabic{enumii}.}
  \tightlist
  \item
    One query to the Root DNS server.
  \item
    One query to the TLD DNS server.
  \item
    One query to the Authoritative DNS server for the domain.
  \end{enumerate}
\end{enumerate}

So, the minimum number of DNS queries would be four if we consider the
chain of queries that the local DNS server has to make on behalf of your
initial query.

\hypertarget{types-of-servers}{%
\subsubsection{Types of Servers}\label{types-of-servers}}

\begin{itemize}
\tightlist
\item
  Local DNS Server (Recursive Resolver)
\item
  Root DNS Server
\item
  TLD (Top-Level Domain) DNS Server
\item
  Authoritative DNS Server
\end{itemize}

In summary, a minimum of four DNS queries will be issued, and they will
be sent to these types of servers. Note that this is a simplified
explanation and actual DNS resolution may involve more complex
scenarios, but this should give you a good basic understanding.

\hypertarget{section-5}{%
\subsection{6}\label{section-5}}

\hypertarget{a-http-version}{%
\subsubsection{(a) HTTP Version}\label{a-http-version}}

HTTP1.1

\hypertarget{b-client-ip-address}{%
\subsubsection{(b) Client ip address}\label{b-client-ip-address}}

10.6.15.132

\hypertarget{c-server-ip-address}{%
\subsubsection{\texorpdfstring{\(c\) Server ip
address}{c Server ip address}}\label{c-server-ip-address}}

128.59.105.24

\hypertarget{d-parallel-connections}{%
\subsubsection{(d) Parallel Connections}\label{d-parallel-connections}}

My browser do open multiple parallel connections, as many TCP handshakes
happening around the same time.

\hypertarget{e-total-number-of-connections}{%
\subsubsection{(e) Total Number Of
Connections}\label{e-total-number-of-connections}}

\hypertarget{f-persistent-connections}{%
\subsubsection{(f) Persistent
Connections}\label{f-persistent-connections}}

Persistent connections in HTTP/1.1 are indicated by the header
\texttt{Connection:\ keep-alive}. Look for this in the \texttt{GET}
request and the server's response. However, there're no such header, so
not persistent connections.

\end{document}
